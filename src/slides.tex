\documentclass[12pt]{beamer}
%\documentclass[20pt,handout]{beamer}
\usetheme{Darmstadt}
\usepackage{graphicx}
\usepackage[german]{babel}
\usepackage[T1]{fontenc}
\usepackage[utf8]{inputenc}
\usepackage{tikz}
\setbeamertemplate{footline}[frame number]

\newcommand{\cc}[1]{\includegraphics[height=4mm]{img/#1.png}}
\usepackage{ifthen}
\newcommand{\license}[2][]{\\#2\ifthenelse{\equal{#1}{}}{}{\\\scriptsize\url{#1}}}
\usepackage{textcomp}

\pgfdeclareimage[height=.6cm]{c3d2logo}{./img/c3d2.pdf} 


\pgfdeclarelayer{foreground}
\pgfsetlayers{main,foreground}
\logo{\pgfputat{\pgfxy(-1,0)}{\pgfbox[center,base]{\pgfuseimage{c3d2logo}}}}


\title{Chaos macht Schule}
\author{\small Marius Melzer \& Stephan Thamm\\\large Chaos Computer Club Dresden}
\date{21.03.2013}

\begin{document}
\maketitle

\section{Einleitung}
\subsection{}

\begin{frame}
  \frametitle{Wer sind wir?}
  \begin{figure}
    \includegraphics[height=0.7\textheight]{img/fingerabdruck.jpg}
  \end{figure}
\end{frame}

\begin{frame}
  \frametitle{Wer sind wir?}
  \begin{figure}
    \includegraphics[height=0.7\textheight]{img/trojaner.jpg}
  \end{figure}
\end{frame}

\begin{frame}
    \frametitle{Wer sind wir?}
    \begin{itemize}
      \item<1-> Chaos Computer Club Dresden (\url{http://c3d2.de})
          \note{}
      \item<2-> Datenspuren: 7. und 8. September 2013 \url{http://datenspuren.de}
      \item<3-> Podcasts (\url{http://pentamedia.de})
      \item<4-> Chaos macht Schule
    \end{itemize}
\end{frame}

\section{Internet}
\subsection{}

\begin{frame}
  \frametitle{Das Internet}
  \begin{figure}
    \includegraphics[height=0.7\textheight]{img/internetstruktur-1.png}
  \end{figure}
\end{frame}

\begin{frame}
  \frametitle{Das Internet}
  \begin{figure}
    \includegraphics[height=0.7\textheight]{img/internetstruktur-2.png}
  \end{figure}
\end{frame}

\begin{frame}
  \frametitle{Das Internet}
  \begin{figure}
    \includegraphics[height=0.7\textheight]{img/internetstruktur-3.png}
  \end{figure}
\end{frame}

\begin{frame}
  \frametitle{Das Internet}
  \begin{figure}
    \includegraphics[height=0.7\textheight]{img/internetstruktur-4.png}
  \end{figure}
\end{frame}

\begin{frame}
  \frametitle{Das Internet}
  \begin{figure}
    \includegraphics[height=0.7\textheight]{img/internetstruktur-5.png}
  \end{figure}
\end{frame}

\begin{frame}
  \frametitle{OSI Schichten}
  \begin{figure}
    \includegraphics[height=0.7\textheight]{img/schichten.png}
  \end{figure}
\end{frame}

\begin{frame}
  \frametitle{HTTP}
  \begin{figure}
    \includegraphics[height=0.7\textheight]{img/http.png}
  \end{figure}
\end{frame}

\begin{frame}
  \frametitle{SMTP}
  \begin{figure}
    \includegraphics[height=0.7\textheight]{img/smtp.jpg}
  \end{figure}
\end{frame}

\section{Kryptographie}
\subsection{}

\begin{frame}
\frametitle{Symmetrische Verschlüsselung}
\begin{figure}
  \includegraphics[height=0.7\textheight]{img/verschl_sym.png}
\end{figure}
\end{frame}

\begin{frame}
\frametitle{Asymmetrische Verschlüsselung}
\begin{figure}
  \includegraphics[height=0.7\textheight]{img/verschl_asym.png}
\end{figure}
\end{frame}

\begin{frame}
\frametitle{Signieren}
\begin{center}\Large
Signieren
\end{center}
\end{frame}

\begin{frame}
\frametitle{Wem Vertrauen?}
\begin{itemize}[<+->]
  \item Zertifikate
  \item "Web of Trust"
\end{itemize}
\end{frame}

\begin{frame}
\frametitle{Web of Trust}
\begin{figure}
  \includegraphics[height=0.7\textheight]{img/ssl.png}
\end{figure}
\end{frame}

\begin{frame}
\frametitle{Web of Trust}
\begin{figure}
  \includegraphics[height=0.7\textheight]{img/weboftrust.png}
\end{figure}
\end{frame}

\section{Anonymität}
\subsection{}

\begin{frame}
  \frametitle{TOR}
  \begin{figure}
    \includegraphics[height=0.7\textheight]{img/tor-1.png}
  \end{figure}
\end{frame}

\begin{frame}
  \frametitle{TOR}
  \begin{figure}
    \includegraphics[height=0.7\textheight]{img/tor-2.png}
  \end{figure}
\end{frame}

\begin{frame}
  \frametitle{TOR}
  \begin{figure}
    \includegraphics[height=0.7\textheight]{img/tor-3.png}
  \end{figure}
\end{frame}

\end{document}
